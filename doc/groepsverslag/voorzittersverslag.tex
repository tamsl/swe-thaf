\documentclass[a4paper,10pt]{article}
\usepackage{graphicx}
\usepackage[dutch]{babel}

\title{Software Engineering en Gedistribueerde Applicaties \\[10pt]Voorzittersverslag\\[25pt]Team F.A.C.H.T.}
\author{\\Chris Bovenschen, Harm Dermois, \\Alexandra Moraga Pizarro, Tamara Ockhuijsen en Fredo Tan \\[10pt]6104096, 0527963, 6129544, 6060374, 6132421 \\[25pt]Universiteit van Amsterdam}
\begin{document}
\maketitle
\newpage
\section{Overzicht}
Als voorzitter van groep FACHT wist ik dat we niet zo goed zijn in programmeren. Daarom heb ik ervoor gekozen alles zo makkelijk mogelijk te
houden. Dit is tot op zekere hoogte gelukt. We hebben eerst alles lokaal 
gedraait en in de laatste 2 weken hebben we alles wat we tot dan hadden
samengevoegd  om een gedistribueerde systeem te maken. We hebben we voor
gekozen geen GUI te maken om dat geen van ons ervaring had met het maken ervan, maar ook wegens tijd gebrek. Ons plan was om aan het eind van week drie
alles klaar te hebben, maar dit was niet gelukt. Toch ben ik blij dat we het meeste werkend hebben gekregen. 
Democratisch hadden we besloten dat we het in python zouden doen. Dit omdat de rest van de groep hun laatste opdracht in python hadden gedaan en het hun 
makkelijker leek. Ik zelf had nog nooit echt iets groots geprogrammeerd in python, maar het bleek dat de rest ook niet veel ervaring hadden op dit gebied.
Dit project heeft ons dus wel wat meer geleerd over python. Python maakt veel dingen makkelijk dus in dat opzicht is het wel een goede taal, maar uiteindelijk geef ik toch mijn voorkeur aan java, omdat je wat meer controle
hebt, maar toch een grote bibliotheek hebt. Ook bleek dat er veel van de python
bibliotheken niet aangeroepen konden worden bijvoorbeeld: numpy/pylab.
\section{Conclusie}
Het waren vier zware weken en we hebben gedaan wat we konden. We gingen voor de
voldoende en ik hoop dat we dat hebben gehaald. We hebben het zo simpel 
mogelijk gehouden en hebben de meeste van onze doelen bereikt.


\end{document}