\documentclass[a4paper,10pt]{article}
\usepackage{graphicx}
\usepackage[dutch]{babel}

\title{Software Engineering en Gedistribueerde Applicaties \\[10pt]Voorzittersverslag\\[25pt]Team F.A.C.H.T.}
\author{\\Chris Bovenschen, Harm Dermois, \\Alexandra Moraga Pizarro, Tamara Ockhuijsen en Fredo Tan \\[10pt]6104096, 0527963, 6129544, 6060374, 6132421 \\[25pt]Universiteit van Amsterdam}
\begin{document}
\maketitle
\newpage
\section{Planning}
De planning die we hadden gemaakt was strikt. Deze planning hebben we geprobeerd aan te houden. Door wat complicaties is het niet gelukt om de planning volledig te volgen. Er waren veel kleine dingen die we over het hoofd hadden gezien. We moesten vaak terug kijken naar oude modules waarvan we dachten dat die al af waren. Ook hadden we maar \'{e}\'{e}n week de tijd gepland voor de integratie, maar om te intregreren zouden alle modules aangepast moeten worden om ze met elkaar te laten communiceren.
\section{Werkwijze}
We werkten in paren om de modules te maken. We hadden besprekingen, maar we namen niet de tijd om alle code gezamelijk door te nemen, dus was er aan het einde een probleem. Dit zorgde ervoor dat niet iedereen de laatste twee weken tegelijk kon werken aan wat er in de planning stond.
\section{Vertrouwen}
Dit team bestaat uit twee groepen die al langer met elkaar hebben samengewerkt. Dit had als gevolg dat er weinig bekend was over wat iedereen kon. We hadden nooit eerder samengewerkt, maar er was natuurlijk wel al het \'{e}\'{e}n ander bekend van elkaar. Het ging niet altijd over rozen.
\section{Keuzes}
We hebben ervoor gekozen om geen GUI te maken, omdat geen van ons ervaring had met het maken van een GUI. Ook leek het ons niet zo een belangrijk deel van de opdracht. Er was niet echt een goed gedefineerde opdracht. 
We besloten het project geheel in Python te doen. Chris en ik hadden hier niet veel mee gewerkt, dus waren er hier en daar wel problemen met het gebruik.
Alle modules hebben we eerst lokaal gedraaid en direct aan de server gedraaid om de werking te testen. Pas wanneer deze klaar waren, werden ze ge\"{i}ntegreerd.
We hebben geprobeerd de logica zo simpel mogelijk te houden, omdat dit meestal het beste werkt.
\section{Conclusie}
Het waren vier zware weken en we hebben gedaan wat we konden.
Dit project heeft ons dus wel wat meer geleerd over Python. Python maakt veel dingen makkelijk, dus in dat opzicht is het wel een goede taal. Uiteindelijk geef ik toch mijn voorkeur aan Java, omdat je dan wat meer controle hebt. Ook bleek dat er veel van de Python libraries niet aanwezig te zijn op de computers.
Het is moeilijk om met mensen waar je nog nooit mee hebt samengewerkt een project goed te door lopen. Het is niet makkelijk om een project te leiden over een onderwerp waar je niet veel van weet, met mensen waarvan je niet precies weet wat ze wel en niet kunnen.
\end{document}